\documentclass[150]{HSMW-Thesis}

%Präambel

\usepackage[utf8]{inputenc} 
\usepackage{ngerman}

\Art{Bachelorarbeit}
\Anrede{Herr}
\Vorname{Thomas}
\Nachname{Kleebaum-Nagy}

\Thema{Auswirkungen innovativer mediengestützter Lehrformen auf das klassische Hochschulstudium}
\Unterthema{}

\Studiengang{Medieninformatik}
\Seminargruppe{Mi12w1-B}
\Fakultaet{Angewandte Computer- und Biowissenschaften}

\Erstpruefer{Prof. Dr. re. oec. Volker Tolkmitt}
\Zweitpruefer{Prof. Alexander Marbach}

\Datum{}

\Tag{}
\Monat{}
\Jahr{}

\Anlagen{}
\Copyright{}
\Textsatz{}
\Druck{}
\Verlag{}
\ISBN{}

\begin{document}
\bibliographystyle{apalike}


\begin{Referat}
% Referat
\end{Referat}

\begin{Vorwort}
% Vorwort
\end{Vorwort}

\Hauptteil
% Diese Anweisung nicht loeschen!

Das ist ein schöner Text  \footnote{\cite{Schulmeister2013}}

\chapter{Fragen}
Berücksichtigen die derzeit verfügbaren MOOC's die aktuellen Erkenntnisse der Mediendidaktik und haben diese gegenüber der klassischen Hochschulausbildung Vorteile?

Ergänzen Sie die klassische Hochschulausbildung oder ersetzen sie Sie?

Welchen Einfluss hat der moderne Medieneinsatz auf die Studienarbeit?

Welchen Einfluss hat der moderne Studieneinsatz auf die zu entwickelnden Kompetenzen?

\chapter{Einleitung}

\chapter{Das klassische Hochschulstudium}

\chapter{Analyse klassischer und moderner Ausbildungsformen}

\chapter{Veränderung der Lern- und Ausbildungsform durch moderne Medien}

\chapter{Ableitung von Empfehlungen für mediengestützte Lehrangebote}

%\section{}

\Anhang

%\chapter{}
\bibliography{HSMW-Thesis-Vorlage}

%\begin{thebibliography}{99}
%\bibitem{} 
%\end{thebibliography}

\end{document}

