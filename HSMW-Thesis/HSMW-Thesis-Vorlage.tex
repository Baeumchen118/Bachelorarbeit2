\documentclass[150]{HSMW-Thesis}

%Präambel

\usepackage[utf8]{inputenc} 
\usepackage{ngerman}
\usepackage{nomencl}
\makenomenclature

\Art{Bachelorarbeit}
\Anrede{Herr}
\Vorname{Thomas}
\Nachname{Kleebaum-Nagy}

\Thema{Auswirkungen innovativer mediengestützter Lehrformen auf das klassische Hochschulstudium}
\Unterthema{}

\Studiengang{Medieninformatik}
\Seminargruppe{Mi12w1-B}
\Fakultaet{Angewandte Computer- und Biowissenschaften}

\Erstpruefer{Prof. Dr. re. oec. Volker Tolkmitt}
\Zweitpruefer{Prof. Alexander Marbach}

\Datum{}

\Tag{}
\Monat{}
\Jahr{}

\Anlagen{}
\Copyright{}
\Textsatz{}
\Druck{}
\Verlag{}
\ISBN{}

\begin{document}
\bibliographystyle{apalike}

\begin{Referat}
% Referat
\end{Referat}

\begin{Vorwort}
% Vorwort
\end{Vorwort}

\Hauptteil
% Diese Anweisung nicht loeschen!
\printnomenclature

\nomenclature{PC}{\uline{P}ersonal \uline{C}omputer}
\nomenclature{PC}{\uline{P}ersonal \uline{C}omputer}

\chapter{Thesenerarbeitung}
Möchte ich didaktisch gute Ausbildung betreiben, muss ich genau wissen mit wehm ich es zu tun habe. Das Ausbildungsmittel muss darauf zugeschnitten sein. 
Ein gutes Ausbildungsmittel für alle gibt es nicht. Was haben die Studenten für eine Vorbildung (Unter- Übervorderung)? 
Besitzen Sie die Fähigkeiten um mit dem Ausbildungsmittel umzugehen. Besitzen Sie die Fähigkeit im den Lernstoff in einer Gruppe erarbeiten zu können?
Müssen Sie evtl. angeleitet/geleitet werden (Blenend learning, Flipped classroom). Verstehen Sie die Sprache? Können Sie die notwendigen Werkzeuge einsetzen.
Muss ich Grundlagen selber lehren oder reicht es aus in der Informationsflut des Internets Empfehlungen zu Tutorien zu geben (P/Q Formel Google mit Ungefähr 102.000 Suchergebnisse)?
Onlineausbildung im Bereich der Grundlagenausbildung oder Spezialisirung? Woher kommen die Studenten (Abitur, 2. Bildungsweg) NC ja nein? Wie ist die Informationstechnische Vorbildung? 
MOCCs als Erscheinung/Konsequenz teurer Studienengebühren? 

Was wollen Bildungsträger? Hohe Einschreibelisten oder hohe Abschlussquoten? Können spezialisierte Lehrformen die Atraktivität der akademischen Schulbildung im Bereich der bundesländer begünstigen? 
Was wolle/fordern zukünftige Arbeitgeber? Was wollen Studenten?



\chapter{Fragen}
\textbf{Berücksichtigen die derzeit verfügbaren MOOC's die aktuellen Erkenntnisse der Mediendidaktik und haben diese gegenüber der klassischen Hochschulausbildung Vorteile?}

\textbf{Ergänzen Sie die klassische Hochschulausbildung oder ersetzen sie Sie?}


\begin{quote}
Um  die  Chancen  und  Potenziale,  aber  auch  die  Schwächen  und  Risiken  dieses Lehr-/Lernszenarios fundierter bewerten zu können, bedarf es aber einer differenzierten 
Betrachtung, wie sie beispielhaft Rolf Schulmeister bereits Ende 2012 in seiner Keynote zur  Campus  Innovation  vorgenommen  hat. \footnote{\cite[S. 7 ff.]{Schulmeister2013} } \footnote{\cite{digitalkompakt07}}
\end{quote}

\begin{quote}
So haben MOOCs in der Regel mit sehr hohen Abbrecherquoten zu kämpfen, die sich vor allem durch die niederschwelligen Einstiegsmöglichkeiten, 
aber auch häufig durch das didaktische Design erklären lassen. Darüber hinaus ist die Frage der Zertifizierung und vor allem der Anerkennung von in MOOCs erbrachten Leistungen 
noch weitestgehend ungeklärt. 

... 

Zudem ist bei vielen Geschäftsmodellen derprofit- als auch non-profit-getriebenen Portalanbieter der Aspekt des Datenschutzes 
im Umgang mit den Leistungsdaten der Kursteilnehmenden rechtlich sehr problematisch.
\end{quote} 

\textbf{Welchen Einfluss hat der moderne Medieneinsatz auf die Studienarbeit?}

\textbf{Welchen Einfluss hat der moderne Studieneinsatz auf die zu entwickelnden Kompetenzen?}

\chapter{Einleitung}

\chapter{Das klassische Hochschulstudium}

\chapter{Analyse klassischer und moderner Ausbildungsformen}

\chapter{Veränderung der Lern- und Ausbildungsform durch moderne Medien}

\chapter{Ableitung von Empfehlungen für mediengestützte Lehrangebote}

%\section{}

\nomenclature[prefix]{symbol}{description}

\Anhang

%\chapter{}
\bibliography{HSMW-Thesis-Vorlage}

%\begin{thebibliography}{99}
%\bibitem{} 
%\end{thebibliography}

\end{document}

